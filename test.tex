%\documentclass{scrartcl}
%\usepackage{lipsum}
%\usepackage{color}
%\usepackage{mdframed,tikz}
%
%\begin{document}
%
%%\begin{mdframed}[backgroundcolor=gray!7,roundcorner=8pt, outerlinewidth=1cm, outerlinecolor=red]
%%\lipsum
%%\end{mdframed}
%
%\newmdtheoremenv[%
%outerlinewidth= 50,%
%roundcorner= 10 pt,%
%leftmargin= 40,%
%rightmargin= 40,%
%backgroundcolor=  yellow!40,%
%outerlinecolor= red,%
%innertopmargin=\topskip,%
%splittopskip=\topskip,%
%ntheorem=  true,%
%]{theorem}{Theorem}[section]
%\begin{theorem}[Pythagoreantheorem]
%\lipsum
%\end{theorem}
%
%\end{document}

\documentclass{article}
\usepackage{lipsum}
\usepackage{xcolor}
\usepackage{mdframed}
\usepackage{hyperref}
\usepackage{marginnote}[fulladjus]

\newmdtheoremenv[%
  backgroundcolor=black!7,
  linecolor=red,
  linewidth=1pt,
  ]{theorem}{Theorem}
  
\reversemarginpar
\renewcommand{\marginnotevadjust}{-.2cm}

\newenvironment{Aufgabe}{
\begin{theorem}
\marginnote{
\begin{Form}
    \CheckBox[width=.6cm]{~}
\end{Form}
}
}{\end{theorem}}

\begin{document}
%\begin{theorem}
%\marginnote{
%\begin{Form}
%    \CheckBox[width=.6cm]{~}
%\end{Form}
%}
\begin{Aufgabe}
\lipsum
\end{Aufgabe}
%\end{theorem}



%Why does Mendel's 2$\mathrm{^{nd}}$ require viability of offspring?}


\end{document} 