\section*{Legende}

\textcolor{blue}{
Blaue Boxen symbolisieren Aufgaben.
Hier sollst du aktiv werden und die Aufgaben selb\-ständig oder mit einem Partner lösen.
Die meisten Aufgaben haben eine Lösung, die du im Anhang einsehen kannst.
Aufgaben haben eine Checkbox. Wenn du die Aufgabe bearbeitet hast, kannst du diese abhaken (per klick).
Speicherst du das Dokument ab, dann weißt du das nächste mal wieder, welche Aufgaben du bereits gelöst hast.
}

\begin{AufgabeON}
\marginnote{
\begin{Form}
    \CheckBox[width=.6cm,name=legende]{~}
\end{Form}
}
Dies ist eine Aufgabe!
\end{AufgabeON}

\textcolor{green}{
Eine grüne Box symbolisiert eine Sicherung.
Diese Informationen solltest du dir in dein Heft übernehmen.
}

\begin{sich}
Dies ist eine Sicherung!
\end{sich}

\textcolor{red}{
Eine rote Box symbolisiert Zusatzwissen.
Zusatzwissen ist nicht prüfungsrelevant, aber sehr interessant.
Zusatzwissen solltest du nur bearbeiten, wenn du gut in der Zeit liegst.
}

\begin{ZW}
Dies ist sehr interessantes Zusatzwissen!
\end{ZW}

Wenn du einen guten PDF-Viewer benutzt, werden Verlinkungen innerhalb des Dokumentes angezeigt. So kannst du durch Klick auf einen Link zu den Aufgaben und Lösungen springen, oder vom Inhaltsverzeichnis  zu einem Kapitel.

Empfohlen wird der PDF-Viewer \textsc{SumatraPDF}.